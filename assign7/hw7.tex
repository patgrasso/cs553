% Created 2016-12-19 Mon 19:59
% Intended LaTeX compiler: pdflatex
\documentclass[11pt]{article}
\usepackage[utf8]{inputenc}
\usepackage[T1]{fontenc}
\usepackage{graphicx}
\usepackage{grffile}
\usepackage{longtable}
\usepackage{wrapfig}
\usepackage{rotating}
\usepackage[normalem]{ulem}
\usepackage{amsmath}
\usepackage{textcomp}
\usepackage{amssymb}
\usepackage{capt-of}
\usepackage{hyperref}
\usepackage[margin=1in]{geometry}
\author{Patrick Grasso}
\date{\today}
\title{Homework 7}
\hypersetup{
 pdfauthor={Patrick Grasso},
 pdftitle={Homework 7},
 pdfkeywords={},
 pdfsubject={},
 pdfcreator={Emacs 24.5.1 (Org mode 9.0.2)}, 
 pdflang={English}}
\begin{document}

\maketitle
\tableofcontents


\section{Chapter 4 Exercises}
\label{sec:orgd854164}
\subsection{1. Validating PageRank}
\label{sec:orgff7bcb0}
\begin{verbatim}
# Patrick Grasso
# PageRank algorithm

# T_j   = docs linking to page A
# C(T)  = links in page T
# PR(A) = d + (1-d)*sum(j; PR(T_j) / C(T_j))

iters   = 25

#           A  B  C
graph   = [[0, 1, 1], # A
           [1, 0, 0], # B
           [1, 0, 0]] # C

ranks   =  [1, 1, 1]

d = 0.1

def prnt(i, arr):
    print(("{:2}, " + ", ".join(["{:.2f}"]*len(ranks))).format(i, *arr))

def PR(page):
    term = lambda other: ranks[other] * graph[page][other] / sum(graph[other])
    return d + (1-d)*sum(map(term, range(len(ranks))))

for i in range(iters + 1):
    prnt(i, ranks)
    for page in range(len(ranks)):
        ranks[page] = PR(page)
\end{verbatim}

\setlength{\parindent}{0cm}
Output from \texttt{validation.py}
\begin{verbatim}
 0, 1.00, 1.00, 1.00
 1, 1.90, 0.96, 0.96
 2, 1.82, 0.92, 0.92
 3, 1.75, 0.89, 0.89
 4, 1.70, 0.87, 0.87
 5, 1.66, 0.85, 0.85
 6, 1.62, 0.83, 0.83
 7, 1.59, 0.82, 0.82
 8, 1.57, 0.81, 0.81
 9, 1.55, 0.80, 0.80
10, 1.54, 0.79, 0.79
11, 1.53, 0.79, 0.79
12, 1.52, 0.78, 0.78
13, 1.51, 0.78, 0.78
14, 1.50, 0.78, 0.78
15, 1.50, 0.77, 0.77
16, 1.49, 0.77, 0.77
17, 1.49, 0.77, 0.77
18, 1.49, 0.77, 0.77
19, 1.48, 0.77, 0.77
20, 1.48, 0.77, 0.77
21, 1.48, 0.77, 0.77
22, 1.48, 0.77, 0.77
23, 1.48, 0.77, 0.77
24, 1.48, 0.76, 0.76
25, 1.48, 0.76, 0.76
\end{verbatim}

\subsection{4. PageRank and internet surfing}
\label{sec:orgc73670b}
PageRank is related to the behavior of internet surfing in that it measures
pages' popularity based on how many references to the page it finds, similar
to how an internet surfer might discover documents by clicking through links
on various pages. Statistically, the more links to a page that exist, the higher
the probability that an internet surfer will click on a link to that page
(generally).

\subsection{5. Most common text}
\label{sec:org3a46f00}
I would guess that the most common anchor text is the title of whichever website
the anchor points to, which is not necessarily helpful for a page ranking
algorithm that relies on such text. There are many instances where people will
simply link the name or title of something, say
\href{https://en.wikipedia.org/wiki/Polar\_bear}{polar bears}, which can be inferred from the title of the page being linked to.

\section{Abiguous Words}
\label{sec:org7347650}
Ambiguous sentences:

\begin{enumerate}
\item Hand me the spoon, please
\item He put his hand on top of his head
\end{enumerate}

The ambiguous word here is "hand". In the first sentence, it is meant to mean
"pass", as in "pass the bread". In the second sentence, "hand" refers to a
human hand, the part of the body attached to the arm.

\begin{verbatim}
$ python ambiguity.py

Ambiguous Word:   hand

Sentence:     hand me the spoon, please
True Defn:    pass.v.05 - place into the hands or custody of
Predict Defn: pass.v.05 - place into the hands or custody of
Similarity:   1.0

Sentence:     he put his hand on top of his head
True Defn:    hand.n.01 - the (prehensile) extremity of the superior limb
Predict Defn: hand.n.08 - a rotating pointer on the face of a timepiece
Similarity:   0.2222222222222222
\end{verbatim}

The first sense of "hand" is matched perfectly. However, the second sentence is
identified with a clock hand. Given the structure of the sentence, this somewhat
makes sense (if some of the words were changed, it might make sense; at least
it is the correct part of speech). However, the use of possessive pronouns like
"his" clearly rules this out as a sensical match.
\end{document}
